\documentclass[a4paper,10pt,addpoints]{exam}

\pagestyle{headandfoot}

\headrule
\header{Stack Builders School of Haskell}{Session 3 Exam}{January 8, 2015}
\footer{}{\thepage}{}

\begin{document}

\begin{center}
  \fbox{\fbox{\parbox{5.5in}{\centering
        The objective of this exam is to test your undertanding of
        weeks 5 and 6 of the CIS 194 Spring 2013 course
        (parametricity, type classes, and lazy evaluation).}}}
\end{center}

\vspace{0.1in}

\makebox[\textwidth]{Name:\enspace\hrulefill}

\vspace{0.2in}

\begin{questions}

  %%%%%%%%%%%%%%%%%%%%%%%%%%%%%%%%%%%%%%%%%%%%%%%%%%%%%%%%%%%%%%%%%%%%

  \question

  Consider the following function:

  \begin{verbatim}
    sd :: Eq b => (a -> b) -> a -> b -> Bool
    sd = ?
  \end{verbatim}

  How would you define \texttt{sd} taking into account its type
  signature?

  \begin{verbatim}
    sd f x y =
  \end{verbatim}

  %%%%%%%%%%%%%%%%%%%%%%%%%%%%%%%%%%%%%%%%%%%%%%%%%%%%%%%%%%%%%%%%%%%%

  \question

  Consider the following function:

  \begin{verbatim}
    g :: ?
    g x y = x + 1 > y
  \end{verbatim}

  What is the type of \texttt{g}?

  \vspace{0.15in}

  \begin{choices}
    \choice \texttt{Ord a => a -> a -> Bool}
    \choice \texttt{(Eq a, Num a) => a -> a -> Bool}
    \choice \texttt{(Eq a, Ord a) => a -> a -> Bool}
    \choice \texttt{(Num a, Ord a) => a -> a -> Bool}
  \end{choices}

  %%%%%%%%%%%%%%%%%%%%%%%%%%%%%%%%%%%%%%%%%%%%%%%%%%%%%%%%%%%%%%%%%%%%

  \question

  Given the following functions:

  \begin{verbatim}
    iterate :: (a -> a) -> a -> [a]
    iterate f x = x : iterate f (f x)

    take :: Int -> [a] -> [a]
    take n _      | n <= 0 = []
    take _ []              = []
    take n (x:xs)          = x : take (n - 1) xs
  \end{verbatim}

  Can you complete the step-by-step evaluation of \texttt{take 3
    (iterate (+ 1) 0)}?

  \begin{verbatim}
      take 3 (iterate (+ 1) 0)
    = take 3 (0 : iterate (+ 1) (0 + 1))
    = 0 : take (3 - 1) (iterate (+ 1) (0 + 1))
    = 0 : take 2 (iterate (+ 1) (0 + 1))
    =
  \end{verbatim}

  \vspace{\stretch{1.5}}

  %%%%%%%%%%%%%%%%%%%%%%%%%%%%%%%%%%%%%%%%%%%%%%%%%%%%%%%%%%%%%%%%%%%%

  \newpage

  \question

  A stream represents a list that must be infinite:

  \begin{verbatim}
    data Stream a = Cons a (Stream a) deriving Show
  \end{verbatim}

  We can define the ruler function
  $$0, 1, 0, 2, 0, 1, 0, 3, 0, 1, 0, 2, 0, 1, 0, 4, 0, 1, 0, 2, ...$$
  as follows:

  \begin{verbatim}
    ruler :: Stream Integer
    ruler = ruler' 0
      where
        ruler' x = interleave (repeat x) (ruler' (x + 1))
  \end{verbatim}

  Here, the \texttt{repeat} function is defined as:

  \begin{verbatim}
    repeat :: a -> Stream a
    repeat x = Cons x (repeat x)
  \end{verbatim}

  And \texttt{interleave} could be defined as either:

  \begin{verbatim}
    interleaveA :: Stream a -> Stream a -> Stream a
    interleaveA (Cons x xs) (Cons y ys) = Cons x (Cons y (interleave ys xs))
  \end{verbatim}

  Or:

  \begin{verbatim}
    interleaveB :: Stream a -> Stream a -> Stream a
    interleaveB (Cons x xs) ys = Cons x (interleave ys xs)
  \end{verbatim}

  What is the difference between using \texttt{interleaveA} or
  \texttt{interleaveB} in \texttt{ruler}?

  \vspace{\stretch{1}}

\end{questions}

\newpage

\end{document}
