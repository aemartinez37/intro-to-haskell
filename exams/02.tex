\documentclass[a4paper,10pt,addpoints]{exam}

\pagestyle{headandfoot}

\headrule
\header{Stack Builders School of Haskell}{Session 5 Exam}{January 22, 2015}
\footer{}{\thepage}{}

\begin{document}

\begin{center}
  \fbox{\fbox{\parbox{5.5in}{\centering
        The objective of this exam is to test your understanding of
        weeks 9 and 10 of the CIS 194 Spring 2013 course (functors and
        applicative functors).
  }}}
\end{center}

\vspace{0.1in}

\makebox[\textwidth]{Name:\enspace\hrulefill}

\vspace{0.2in}

The \texttt{Maybe} type encapsulates an optional value:

\begin{verbatim}
data Maybe a = Nothing | Just a deriving Show
\end{verbatim}

The \texttt{Functor} type class is used for types that can be mapped
over:

\begin{verbatim}
class Functor f where                             instance Functor Maybe where
  fmap :: (a -> b) -> f a -> f b                    fmap _ Nothing  = Nothing
                                                    fmap f (Just x) = Just (f x)
\end{verbatim}

Instances of \texttt{Functor} should satisfy the following laws:

\begin{itemize}
\item
  \texttt{fmap id == id}
\item
  \texttt{fmap (g .\ f) == fmap g .\ fmap f}
\end{itemize}

\texttt{(<\$>)} is an infix synonym for \texttt{fmap}:

\begin{verbatim}
(<$>) :: Functor f => (a -> b) -> f a -> f b
(<$>) = fmap
\end{verbatim}

An applicative functor is a functor with application:

\begin{verbatim}
class Functor f => Applicative f where            instance Applicative Maybe where
  pure  :: a -> f a                                 pure = Just
  (<*>) :: f (a -> b) -> f a -> f b                 Nothing <*> _  = Nothing
                                                    Just f  <*> mx = fmap f mx
\end{verbatim}

\begin{questions}

  %%%%%%%%%%%%%%%%%%%%%%%%%%%%%%%%%%%%%%%%%%%%%%%%%%%%%%%%%%%%%%%%%%%%

  \question[1]

  Every \texttt{Applicative} is also a \texttt{Functor}, so we can
  define \texttt{fmap} in terms of \texttt{pure} and
  \texttt{(<*>)}. Try it:

  \begin{verbatim}
    liftA :: Applicative f => (a -> b) -> f a -> f b
    liftA f mx =
  \end{verbatim}

  %%%%%%%%%%%%%%%%%%%%%%%%%%%%%%%%%%%%%%%%%%%%%%%%%%%%%%%%%%%%%%%%%%%%

  \question[1]

  What's so special about applicative functors? Why is an applicative
  functor more powerful than a functor? If you want, you can use the
  \texttt{Maybe} type as an example.

  %%%%%%%%%%%%%%%%%%%%%%%%%%%%%%%%%%%%%%%%%%%%%%%%%%%%%%%%%%%%%%%%%%%%

  \newpage

  \question[1]

  We can represent a book as an author and a title:

  \begin{verbatim}
    data Book = Book String String deriving Show
  \end{verbatim}

  Consider the following expressions:

  \begin{verbatim}
    maybeAuthor1, maybeAuthor2 :: Maybe String
    maybeAuthor1 = Just "Charles Dickens"
    maybeAuthor2 = Nothing

    maybeTitle1, maybeTitle2 :: Maybe String
    maybeTitle1 = Nothing
    maybeTitle2 = Just "David Copperfield"
  \end{verbatim}

  Can you match the following expressions with their results?

  \vspace{0.15in}

  \begin{parts}
    \part \texttt{ghci> Book <\$> maybeAuthor1 <*> maybeTitle1} \answerline
    \part \texttt{ghci> Book <\$> maybeAuthor1 <*> maybeTitle2} \answerline
    \part \texttt{ghci> Book <\$> maybeAuthor2 <*> maybeTitle1} \answerline
    \part \texttt{ghci> Book <\$> maybeAuthor2 <*> maybeTitle2} \answerline
  \end{parts}

  For each expression, choose one of the following options:

  \begin{enumerate}
  \item \texttt{Just (Book "David Copperfield" "Charles Dickens")}
  \item \texttt{Nothing}
  \item \texttt{Just (Book "Charles Dickens" "David Copperfield")}
  \item \texttt{Book (Just "Charles Dickens") (Just "David Copperfield")}
  \end{enumerate}

  \bonusquestion[1]

  An \texttt{Alternative} is a monoid on applicative functors:

  \begin{verbatim}
class Applicative f => Alternative f where         instance Alternative Maybe where
  empty :: f a                                       empty = Nothing
  (<|>) :: f a -> f a -> f a                         Nothing <|> mx = mx
                                                     mx      <|> _  = mx
  \end{verbatim}

  What's so special about \texttt{Alternative}? Why is an
  \texttt{Alternative} useful? If you want, you can use the
  \texttt{Maybe} type as an example.

\end{questions}

\newpage

\end{document}
