\documentclass[a4paper,10pt,addpoints]{exam}

\pagestyle{headandfoot}

\headrule
\header{Stack Builders School of Haskell}{Session 6 Exam}{January 29, 2015}
\footer{}{\thepage}{}

\begin{document}

\begin{center}
  \fbox{\fbox{\parbox{5.5in}{\centering
        The objective of this exam is to test your understanding of
        weeks 11 and 12 of the CIS 194 Spring 2013 course (applicative
        functors and monads).
  }}}
\end{center}

\vspace{0.1in}

\makebox[\textwidth]{Name:\enspace\hrulefill}

\vspace{0.2in}

\begin{questions}

  %%%%%%%%%%%%%%%%%%%%%%%%%%%%%%%%%%%%%%%%%%%%%%%%%%%%%%%%%%%%%%%%%%%%

  \bonusquestion[1]

  In week 9, we studied functors, which allow us to map a function
  over some structure:

  \begin{verbatim}
  class Functor f where                       instance Functor Maybe where
    fmap :: (a -> b) -> f a -> f b              fmap _ Nothing  = Nothing
                                                fmap f (Just x) = Just (f x)
  \end{verbatim}

  In weeks 10 and 11, we studied applicative functors, which allow us
  to map a function contained over some structure over some structure:

  \begin{verbatim}
  class Functor f => Applicative f where      instance Applicative Maybe where
    pure  :: a -> f a                           pure = Just

    (<*>) :: f (a -> b) -> f a -> f b           Nothing <*> _  = Nothing
                                                Just f  <*> mx = fmap f mx
  \end{verbatim}

  And, in week 12, we studied monads:

  \begin{verbatim}
  class Applicative m => Monad m where        instance Monad Maybe where
    (>>=) :: m a -> (a -> m b) -> m b           Nothing >>= _ = Nothing
                                                Just x  >>= k = k x
  \end{verbatim}

  Monads are more powerful than functors and applicative functors
  because they allow us to remove a level of structure using the
  \texttt{join} function:

  \begin{verbatim}
    ghci> join Nothing                        ghci> join (Just (Just 1))
    Nothing                                   Just 1
    ghci> join (Just Nothing)                 ghci> join [[1,2],[],[3,4,5]]
    Nothing                                   [1,2,3,4,5]
  \end{verbatim}

  Define \texttt{join}:

  \begin{verbatim}
    join :: Monad m => m (m a) -> m a
    join mmx =
  \end{verbatim}

  (Hint: Use \texttt{id} and remember that \texttt{m} is a
  \texttt{Monad}.)

  %%%%%%%%%%%%%%%%%%%%%%%%%%%%%%%%%%%%%%%%%%%%%%%%%%%%%%%%%%%%%%%%%%%%

  \question[1]

  The bind operator, \texttt{(>>=)}, sequentially composes two
  actions, passing any value produced by the first as an argument to
  the second. Write \texttt{(>>=)} in terms of \texttt{fmap} and
  \texttt{join}:

  \begin{verbatim}
    (>>=) :: Monad m => m a -> (a -> m b) -> m b
    mx >>= k =
  \end{verbatim}

  %%%%%%%%%%%%%%%%%%%%%%%%%%%%%%%%%%%%%%%%%%%%%%%%%%%%%%%%%%%%%%%%%%%%

  \newpage

  \question[1]

  Consider the following function:

  \begin{verbatim}
    mapA :: Applicative f => (a -> f b) -> [a] -> f [b]
    mapA _ []     = pure []
    mapA f (x:xs) = (:) <$> f x <*> mapA f xs
  \end{verbatim}

  What is the result of \texttt{mapA Just [1,2,3]}?

  \begin{choices}
    \choice \texttt{[1,2,3]}
    \choice \texttt{Just [1,2,3]}
    \choice \texttt{[Just 1,Just 2,Just 3]}
    \choice \texttt{Nothing}
  \end{choices}

  \vspace{0.15in}

  \question[2]

  Consider the following function:

  \begin{verbatim}
    sequenceA :: Applicative f => [f a] -> f [a]
    sequenceA = mapA id
  \end{verbatim}

  \begin{parts}

    \part

    What is the result of \texttt{sequenceA [Just 1,Nothing,Just 3]}?

    \begin{choices}
      \choice \texttt{[1,3]}
      \choice \texttt{Just [1,3]}
      \choice \texttt{[Just 1,Just 3]}
      \choice \texttt{Nothing}
    \end{choices}

    \part

    What is the result of \texttt{sequenceA [Just 1,Just 2,Just 3]}?

    \begin{choices}
      \choice \texttt{[1,2,3]}
      \choice \texttt{Just [1,2,3]}
      \choice \texttt{[Just 1,Just 2,Just 3]}
      \choice \texttt{Nothing}
    \end{choices}

  \end{parts}

  \question[1]

  Consider the following functions:

  \begin{verbatim}
    replicate :: Int -> a -> [a]
    replicate n x = take n (repeat x)

    replicateA n fx = sequenceA (replicate n fx)
  \end{verbatim}

  What is the type of \texttt{replicateA}?

  \begin{choices}
    \choice \texttt{Functor f => Int -> f a -> f [a]}
    \choice \texttt{Int -> f a -> f [a]}
    \choice \texttt{Applicative f => Int -> f a -> f [a]}
    \choice \texttt{Applicative f => Int -> f a -> [f a]}
  \end{choices}

\end{questions}

\newpage

\end{document}
